
% Leave a blank line at top to work around some flakiness in l2h
% on RedHat 5.2/AXP (otherwise the images.tex file can get some
% bad characters dumped into it).
%
% The following interlock is taken from Knuth's  ``The TeXbook'',
% p383 (Appendix D: Dirty Tricks).  It insures this file gets read
% at most once.  (NOTE: Unfortunately, latex2html doesn't do
% \if statements.)
%\ifx\oommfheadread\relax\endinput\else\let\oommfheadread=\relax\fi

\documentclass[12pt]{article}

\usepackage{makeidx}

%begin{latexonly}
% Use package pdftex, iff we are running pdftex with pdf output
% (This logic is from Heiko Oberdiek's ifpdf package.)
\ifx\pdfoutput\undefined
  % not running PDFTeX
  \def\oommfpdf{0}
\else
  \ifx\pdfoutput\relax
    % not running PDFTeX
    \def\oommfpdf{0}
  \else
    % running PDFTeX, with...
    \ifnum\pdfoutput>0
      % ...PDF output
      \def\oommfpdf{1}
    \else
      % ...DVI output
      \def\oommfpdf{0}
    \fi
  \fi
\fi
\ifnum\oommfpdf=1
  \usepackage[pdftex, colorlinks=true, citecolor=blue]{hyperref}
\fi
%end{latexonly}

\ifnum\oommfpdf=0
% pdflatex command not in use
% The html package included below uses \pdfoutput to determine whether
% or not pdf-TeX is being used.  Unfortunately, the code in html.sty
% that determines this is broken, at least v1.39 2001/10/01 as shipped
% with Fedora Core 6 (FC6) when used with the latex in FC6.  This
% results in breakage of some commands defined in html.sty, including at
% least \htmlimage and \htmladdnormallink.  The breakage is such that
% the pdf-versions of these commands are wrongly defined in the case
% where latex or latex2html is running.  One workaround would be to
% redefine these commands after \usepackage{html}, but a more general
% fix would appear to be to just redefine \pdfoutput so that the logic
% in html.sty works.  This latter approach is done here, by unsetting
% \pdfoutput.  I've included this lengthy note because the problematic
% case is when \pdfoutput is defined to value 0.  Is there code
% someplace that functions differently if \pdfoutput is 0 than if it is
% undefined?  I don't know.  If it turns out that this breaks something,
% then one can try redefining \pdfoutput to 0 after \usepackage{html},
% or otherwise leaving \pdfoutput alone and just redefining \htmlimage
% etc. as needed.
\let\pdfoutput\relax
\fi

% \ifnum\oommfpdf=0
% pdflatex command not in use
% \renewcommand{\htmlimage}[1]{} % pdf-mode detection code is broken
%    in some versions of html.sty, causing \htmlimage to be re-defined
%    as taking 2 arguments.  This causes some havoc.  Define it back to
%    be safe.
%    NOTE: This code superceded by \let\pdfoutput\relax code above.
%    The code in this current stanza is left in but commented out in
%    case it occurs that the more general fix above breaks stuff.
% \renewcommand{\htmladdnormallink}[2]{#1} % Ditto.
% \fi


%\usepackage[pdftex, colorlinks=true, citecolor=blue]{hyperref}
\usepackage{html}

%begin{latexonly}
\usepackage{../common/oommf} % Base style redefinitions and/or patches
%\ifx\undefined\pdfpagewidth % pdflatex command not in use
\ifnum\oommfpdf=0
% pdflatex command not in use
\newcommand{\pdfonly}[1]{}
\newcommand{\ifnotpdf}[1]{#1}
\usepackage{color}
\usepackage{graphics}
\else                       % pdflatex command in use
\newcommand{\pdfonly}[1]{#1}
\newcommand{\ifnotpdf}[1]{}
\pdfcompresslevel=9
%\pdfpagesattr={/CropBox [60 290 480 720]}
%\pdfpagewidth=6.0in
%\pdfpageheight=5.5in
%\pdfcatalog{/PageMode /UseOutlines}
\pdfcatalog{            % Catalog dictionary of PDF output.
    /PageMode /UseOutlines
    /URI (http://math.nist.gov/oommf/)
}
% openaction goto page 1 {/Fit}

\usepackage[pdftex]{color}
\usepackage[pdftex]{graphics}
\fi
\newcommand{\htmlonlyref}[2]{#1}
%end{latexonly}

\html{
\newcommand{\pdfonly}[1]{}
\newcommand{\ifnotpdf}[1]{#1}
\usepackage{color}
\usepackage{graphics}
%\let\hyperrefhtml=\hyperref
\newcommand{\hyperrefhtml}[3]{\hyperref{#1}{#2}{#3}}
\newcommand{\htmlonlyref}[2]{\htmlref{#1}{#2}}
% Use \htmlonlyref for links to be available in HTML, but not
% in PDF.  In particular, this applies to \label{} commands not
% placed near counter updates, since latex2html drops an anchor
% tag at the right location, but pdflatex just drops the ball
% (well at least pdflatex Version 3.14159-13d (Web2C 7.3.1) does).
}

\setlength{\textwidth}{6.5in}
\setlength{\oddsidemargin}{0in}
\setlength{\textheight}{8.5in}

\begin{htmlonly}
\HTMLset{myTODAY}{\today}
\usepackage{localmods}
\end{htmlonly}

\newcommand{\myimage}[2]{\HTMLcode[#1 #2]{IMG}}

\htmlinfo*
\bodytext{BGCOLOR="#FFFFFF",text="#000000",LINK="#0000FF",
            VLINK="#4498F0",ALINK="00FFFF"}

%\htmladdtonavigation{\htmladdnormallink
%    {\myimage{ALT="OOMMF Home",BORDER="2"}{http://math.nist.gov/oommf/images/oommficon.gif}}{http://math.nist.gov/oommf}}

\htmladdtonavigation{\htmladdnormallink
    {\myimage{ALT="OOMMF Home",BORDER="2"}{oommficon.gif}}{http://math.nist.gov/oommf}}

\newcommand{\blackhole}[1]{}
\newcommand{\Unix}{Unix}
\newcommand{\X}{X}
\newcommand{\Windows}{Windows}
\newcommand{\MacOSX}{Mac OS X}
\newcommand{\DOS}{DOS}
\newcommand{\Tcl}{Tcl}
\newcommand{\C}{C}
\newcommand{\Cplusplus}{C++}
\newcommand{\Tk}{Tk}
\newcommand{\OOMMF}{OOMMF}
\newcommand{\MIF}{MIF}
\newcommand{\ODT}{ODT}
\newcommand{\OVF}{OVF}
\newcommand{\SVF}{SVF}
\newcommand{\VIO}{VIO}
\newcommand{\OBS}{OBS}
\newcommand{\eps}{Encapsulated PostScript}
\newcommand{\postscript}{PostScript}
\newcommand{\mumag}{\latex{$\mu$}\html{mu}MAG}
\newcommand{\micrometer}{\latex{$\mu$m}\html{\begin{rawhtml}&micro;m\end{rawhtml}}}
\newcommand{\munaught}{\latex{$\mu_0$}\html{\begin{rawhtml}&micro;<SUB>0</SUB>\end{rawhtml}}}
\newcommand{\SI}{SI}     % as in SI units
\newcommand{\ASCII}{ASCII}

\newcommand{\oxslabel}[1]{\textbf{#1}}
% Use \oxslabel to refer to Oxs Specify block labels the first time
% in the running text.

\newcommand{\oxsval}[1]{\textit{\textrm{#1}}}
% Use \oxsval for the value portion of label+value keys in both
% the TeX version of the Specify block, and in the running text.

% Filenames and program code identifiers
\blackhole{
\definecolor{fn}{rgb}{0,0.5,0}
\definecolor{cd}{rgb}{0.5,0,0}
\definecolor{btn}{rgb}{0.5,0,0}
\newcommand{\fn}[1]{\latex{{\tt #1}}\html{\textcolor{fn}{#1}}}   % Files
\newcommand{\cd}[1]{\latex{{\tt #1}}\html{\textcolor{cd}{#1}}}   % Code
\newcommand{\btn}[1]{\latex{{\tt #1}}\html{\textcolor{btn}{#1}}} % Buttons
} % blackhole

%begin{latexonly}
\newcommand{\bftt}[1]{\textsf{\textbf{#1}}}
%% This is meant to be a bold tt, but there is no boldface cmtt (TeX
%% Typewriter font)!
\newcommand{\app}[1]{\textbf{#1}}    % Apps
\newcommand{\key}[1]{\texttt{#1}}    % Keys
\newcommand{\fn}[1]{\texttt{#1}}     % Files
\newcommand{\cd}[1]{\texttt{#1}}     % Code
\newcommand{\btn}[1]{\bftt{#1}}      % Buttons
\newcommand{\wndw}[1]{\textbf{#1}}   % Windows
%end{latexonly} % Close \latex
\begin{htmlonly}
\newcommand{\bftt}[1]{\texttt{\textbf{#1}}}
%% NOTE: There is no boldface cmtt (TeX), but HTML browsers
%%  may render differently?!
\newcommand{\app}[1]{\textbf{#1}}    % Apps
\newcommand{\key}[1]{\bftt{#1}} % Keys
\newcommand{\fn}[1]{\bftt{#1}}  % Files
\newcommand{\cd}[1]{\texttt{#1}}  % Code
\definecolor{btn}{rgb}{0.5,0,0}          % Buttons
\newcommand{\btn}[1]{{\textcolor{btn}{\textbf{#1}}}}
\newcommand{\wndw}[1]{{\bf #1}} % Windows
\end{htmlonly} % close \html


% Latex2html inserts unwanted whitespace after \rm, in structures like
%      \newcommand{\vB}{{\rm\bf B}}
% but the following seem to work:
\newcommand{\vB}{\textbf{B}}
\newcommand{\vH}{\textbf{H}}
\newcommand{\vM}{\textbf{M}}
\newcommand{\vm}{\textbf{m}}
\newcommand{\vh}{\textbf{h}}
\newcommand{\vx}{\textbf{x}}

\newcommand{\lb}{\texttt{\#}}  % "Pound" symbol
\newcommand{\pipe}{\latex{{\tt|}}\html{|}} % "Pipe" symbol
\newcommand{\bs}{\texttt{\char'134}} % Backslash, tt font
\newcommand{\fs}{\texttt{/}} % Forward slash, tt font

% MIF 2.x Specify block definitions.
\newcommand{\bi}{\hspace*{2em}}
% \bi is bullet indent.
\newcommand{\ocb}{\textrm{\{}}
\newcommand{\ccb}{\textrm{\}}}
% \ocb is open curly brace, \ccb is close curly brace.

% Bold open and close angle brackets (aka less-than and greater-than
% symbols)
\newcommand{\boa}{\latex{{\boldmath$<$}}\html{\texttt{\textbf{<}}}}
\newcommand{\bca}{\latex{{\boldmath$>$}}\html{\texttt{\textbf{>}}}}

% ``Launching'' option keyword lists font selection
\newcommand{\optkey}[1]{\latex{\textbf{#1}}\html{\texttt{\textbf{#1}}}}

% Codelisting environment
%begin{latexonly}
\newenvironment{codelisting}[4]{%
 \def\codelistingtype{#1}     % f for float, p for ``in page''
 \def\codelistinglabel{#2}    % \label tag
 \def\codelistingcaption{#3}  % caption
 % Parameter #4 is used in the \html version, and is a \ref-style
 % tag to the location in the main text describing the codelisting.
 \if\codelistingtype f \begin{figure}
 \fi
}{
 \if\codelistingtype f  
   \caption{\codelistingcaption\label{\codelistinglabel}}\end{figure}
 \else
    \nopagebreak\parbox{\textwidth}{
    \begin{center}
    \refstepcounter{figure}
    Figure \thefigure: {\codelistingcaption\label{\codelistinglabel}}
    \end{center}
   }\pagebreak[2]
 \fi
}
%end{latexonly}

\html{
\newenvironment{codelisting}[4]{%
  \addtocounter{figure}{1}\label{#2}
  \HTMLsetenv{codelistingcaption}{#3}
  \HTMLsetenv{textlink}{#4}
  \htmlrule
}{
  \begin{center}
  Figure \thefigure:
    \HTMLget{codelistingcaption}
    \htmlref{(Description.)}{\HTMLget{textlink}}
  \end{center}
  \htmlrule
}}


% Ersatz figure environment.  This is a standard figure environment in
% LaTex, but a dummy block in HTML.  This is useful because LaTeX2HTML
% passes figure environments to LaTex, and converts the resulting
% postscript to a graphics bitmap for inclusion.  Sometimes we don't
% want this, for example if the figure data is already in bitmap format.
% Also, we may want to throw in an ALT tag.
% SAMPLE USAGE:
%   \ofig{\includeimage{6in}{!}{oxsclass}{OXS top-level class
%         diagram}}{OXS top-level class diagram.}{fig:oxsclass}
%
\latex{
\newcommand{\ofig}[3]{%
\begin{figure}
 \begin{center}
   #1\\
   \caption{#2\label{#3}}
 \end{center}
\end{figure}}
}
\html{
\newcommand{\ofig}[3]%
{\begin{center}
 \addtocounter{figure}{1}\label{#3}
 \textbf{Figure \thefigure: #2}\\
 #1
\end{center}}
}
%% Is \refstepcounter{figure} needed in the \html def?


% Graphics inclusion.
%  Usage: \includepic{basename}{altstring}
%     A fixed scale parameter is used in the LaTeX code;
%   under HTML the graphic is brought directly in without any scaling.
%     Basename is the name of the graphic to include,
%   expanded as psfiles/basename.ps under latex, and
%   giffiles/basename.gif under html.
%     Altstring it a string to be passed to the ALT= tag
%   in HTML.  It is not used in the LaTeX code.
%begin{latexonly}
\newcommand{\includepic}[2]{%
\scalebox{0.3333}{\includegraphics{psfiles/#1.ps}} } \pdfonly{
\renewcommand{\includepic}[2]{%
\scalebox{0.75}{\includegraphics{pngfiles/#1.png}} } }
%end{latexonly}
\begin{htmlonly}
\newcommand{\includepic}[2]{%
\HTMLcode[../giffiles/#1.gif,ALT="#2"]{IMG}
}
\end{htmlonly}

% Alternate graphics inclusion
%  Usage: \includeimage{width}{height}{basename}{altstring}
%     Width and height are dimensions, e.g., 4in.  One of
%   these may be an exclamation mark '!', in which case
%   the corresponding dimension will be scaled as necessary
%   to keep the original aspect ratio.  Presently these two
%   parameters are used only in the LaTeX code; under HTML
%   graphic is brought directly in without any scaling.
%     Basename is the name of the graphic to include,
%   expanded as psfiles/basename.ps under latex, and
%   giffiles/basename.gif under html.
%     Altstring it a string to be passed to the ALT= tag
%   in HTML.  It is not used in the LaTeX code.
%begin{latexonly}
\newcommand{\includeimage}[4]{%
\resizebox{#1}{#2}{\includegraphics{psfiles/#3.ps}}
}
\pdfonly{%
\renewcommand{\includeimage}[4]{%
\resizebox{#1}{#2}{\includegraphics{pngfiles/#3.png}}
}
}
%end{latexonly}
\begin{htmlonly}
\newcommand{\includeimage}[4]{%
\HTMLcode[../giffiles/#3.gif,ALT="#4"]{IMG}
}
\end{htmlonly}

% Work around for some apparently broken LaTeX2HTML Table of Contents
% controls.
\latex{
\def\ssechead{\subsection*}
\def\sssechead{\subsubsection*}
}
\html{
\newcommand{\ssechead}[1]{\par\noindent{\Large\bf{#1}}\\}
\newcommand{\sssechead}[1]{\par\noindent{\large\bf{#1}}\\}
}


% If an inline formula has positive depth, then LaTeX2HTML handles
% vertical positioning of that formula by adding a vertical rule so
% that the depth and height are equal.  The resulting image is then
% marked in the HTML with the align=middle tag, which aligns the
% vertical center of the image with the current baseline.  This adds
% extra whitespace below the image, sometimes a lot, which can yield
% essentially an extra blank line in the viewed HTML.  The \abovemath
% command raises the math-mode formulae just enough so that the depth
% is zero, in which case the generated image is aligned in the HTML
% with the align=bottom tag.  This also looks bad, so it is a matter
% of choice which is the worse evil.  But it is probably an improvement
% in situations with the formula extends just a *little* below the
% baseline.  WRT the TeX output, this command just renders the formula
% in in-line math mode.
\newcommand{\nodepth}[1]{% Auxiliary command
$\mbox{\renewcommand{\arraystretch}{0}%
$\begin{array}[b]{@{}c@{}}#1\\\rule{1pt}{0pt}\end{array}$}$}
\newcommand{\abovemath}[1]{\latex{$#1$}\html{\nodepth{#1}}}

% Hyphenation
\hyphenation{DataTable}

% Index generation
\makeindex

\usepackage{l2hbugs}

%\HTMLset{toppage}{progman.html}
%\htmladdtonavigation{\htmladdnormallink{\htmladdimg{../common/contents.gif}}{progman.html}}

\begin{document}

\nocite{*}  % Include all entries from .bib file.  Putting this at the
	    % top retains the .bib file ordering.


\pagenumbering{roman}
\begin{titlepage}
\label{page:contents}
\par
\vspace*{\fill}
\begin{center}
\Large\bf
\OOMMF\\
Programming Manual\\[2ex]
\large
{\today}
{}\\[2ex]
This manual documents release 1.2a3.\\[1ex]
WARNING: In this alpha release, the
documentation may not be up to date.\\[1ex]
WARNING: This document in under construction.

\end{center}
\vspace{10\baselineskip}
\begin{abstract}
This manual provides source code level information on \OOMMF\ (Object Oriented Micromagnetic Framework),
a public domain micromagnetics program developed at the
\htmladdnormallink{National Institute of Standards and Technology}
{http://www.nist.gov/}.  Refer to the \OOMMF\ User's Guide for an
overview of the project and end-user details.
\end{abstract}
\vspace*{\fill}
\par
\end{titlepage}

\begin{latexonly}
\tableofcontents
\end{latexonly}

% Index cross-references; if these are moved to the bottom of this file
% then two LaTeX passes are required to get them in the .idx file.

\newpage
\section*{Disclaimer}
\addcontentsline{toc}{section}{Disclaimer}

This software was developed at the National Institute of Standards and
Technology by employees of the Federal Government in the course of their
official duties.  Pursuant to Title 17, United States Code, Section 105,
this software is not subject to copyright protection and is in the
public domain\index{license}.

\OOMMF\ is an experimental system.  NIST assumes no responsibility
whatsoever for its use by other parties, and makes no guarantees,
expressed or implied, about its quality, reliability, or any other
characteristic.

We would appreciate acknowledgement if the software is used.  When
referencing \OOMMF\ software, we recommend citing the NIST technical
report, M. J. Donahue and D. G. Porter, ``OOMMF User's Guide, Version
1.0,'' \textbf{NISTIR 6376}, National Institute of Standards and
Technology, Gaithersburg, MD (Sept 1999).

Commercial equipment and software referred to on these pages are
identified for informational purposes only, and does not imply
recommendation of or endorsement by the National Institute of Standards
and Technology, nor does it imply that the products so identified are
necessarily the best available for the purpose.

\newpage

\pagenumbering{arabic}

\section{Overview of \OOMMF}\label{sec:overview}
The goal of the
\htmladdnormallinkfoot{\OOMMF}{http://math.nist.gov/oommf/} (Object
Oriented Micromagnetic Framework) project in the
\htmladdnormallink{Information Technology
Laboratory}{http://www.itl.nist.gov/} (ITL) at the
\htmladdnormallink{National Institute of Standards and
Technology}{http://www.nist.gov/} (NIST)
is to develop a portable, extensible public domain micromagnetic
program and associated tools.  This code will form a completely
functional micromagnetics package, but will also have a well
documented, flexible programmer's interface so that people developing
new code can swap their own code in and out as desired.  The main
contributors to \OOMMF\ are
\ifnotpdf{\htmladdnormallink{Mike Donahue}{http://math.nist.gov/\~{}MDonahue}}
\pdfonly{\htmladdnormallink{Mike Donahue}{http://math.nist.gov/\%7EMDonahue}}
and
\ifnotpdf{\htmladdnormallink{Don Porter}{http://math.nist.gov/\~{}DPorter}.}
\pdfonly{\htmladdnormallink{Don Porter}{http://math.nist.gov/\%7EDPorter}.}

In order to allow a programmer not familiar with the code as a whole
to add modifications and new functionality, we feel that an object
oriented approach is critical, and have settled on C++ as a good
compromise with respect to availability, functionality, and
portability.  In order to allow the code to run on a wide variety of
systems, we are writing the interface and glue code in \Tcl/\Tk.  This
enables our code to operate across a wide range of \Unix\ platforms,
\Windows~NT, and \Windows~9X.

The code may be modified at 3 distinct levels.  At the top level,
individual programs interact via well-defined protocols across network
sockets\index{network~socket}.  One may connect these modules together
in various ways from the user interface, and new modules speaking the
same protocol can be transparently added.  The second level of
modification is at the \Tcl/\Tk\ script level.  Some modules allow
\Tcl/\Tk\ scripts to be imported and executed at run time, and the top
level scripts are relatively easy to modify or replace.  At the lowest
level, the C++ source is provided and can be modified, although at
present the documentation for this is incomplete (cf.\ the ``OOMMF
Programming Manual'').

The first portion of OOMMF released was a magnetization file display
program called
\htmladdnormallink{\app{mmDisp}}{http://math.nist.gov/oommf/mmdisp/mmdisp.html}\index{application!mmDisp}.
A \htmladdnormallinkfoot{working
release}{http://math.nist.gov/oommf/software.html} of the complete OOMMF
project was first released in October, 1998.  It included a problem
editor, a 2D micromagnetic solver\index{simulation~2D}, and several
display widgets, including an updated version of \app{mmDisp}.  The
solver can be controlled by an {\hyperrefhtml{interactive
interface}{interactive interface (Sec.~}{)}{sec:mmsolve2d}}, or through
a sophisticated {\hyperrefhtml{batch control system}{batch control
system (Sec.~}{)}{sec:obs}}.  This solver was originally based on a
micromagnetic code that
\ifnotpdf{\htmladdnormallink{Mike Donahue}{http://math.nist.gov/\~{}MDonahue}}
\pdfonly{\htmladdnormallink{Mike Donahue}{http://math.nist.gov/\%7EMDonahue}}
and
\htmladdnormallink{Bob McMichael}{mailto:rmcmichael@nist.gov}
had previously developed.  It utilizes a Landau-Lifshitz
ODE\index{ODE!Landau-Lifshitz} solver to relax 3D spins on a 2D
mesh\index{grid} of square cells, using FFT's\index{FFT} to compute the
self-magnetostatic (demag) field\index{field!demag}.  Anisotropy,
applied field\index{field!applied}, and initial
magnetization\index{magnetization!initial} can be varied pointwise, and
arbitrarily shaped elements\index{boundary} can be modeled.  

The current development version, \OOMMF\ 1.2, includes Oxs, the
\OOMMF\ eXtensible Solver.  Oxs offers users of \OOMMF\ the ability
to extend Oxs with their own modules.  The details of programming
an Oxs extension module are found in the
\htmladdnormallinkfoot{OOMMF Programming Manual
}{http://math.nist.gov/oommf/doc/}.  The extensible nature of the Oxs
solver means that its capabilities may be varied as necessary for the
problem to be solved.  Oxs modules distributed as part of \OOMMF\
support full 3D\index{simulation~3D} simulations suitable for modeling
layered materials.

{\samepage
If you want to receive e-mail\index{e-mail}
notification\index{announcements} of updates to this project, register
your e-mail address with the ``\mumag\ Announcement'' mailing list:
\begin{center}
\ifnotpdf{\htmladdnormallink{http://www.ctcms.nist.gov/\~{}rdm/email-list.html}{http://www.ctcms.nist.gov/\~{}rdm/email-list.html}.}
\pdfonly{\htmladdnormallink{http://www.ctcms.nist.gov/\~{}rdm/email-list.html}{http://www.ctcms.nist.gov/\%7Erdm/email-list.html}.}
\end{center}
} % end \samepage

The \OOMMF\ developers are always interested in your comments about
\OOMMF.  See the \hyperrefhtml{Credits}{Credits (Sec.~}{)}{sec:credits}
for instructions on how to contact them, and for information on
referencing \OOMMF.

\section{Platform-Independent Make}\label{sec:pimake}

\centerline{\textbf{UNDER CONSTRUCTION}}

Details on pimake go here.

Somewhere we should have documentation on feeding and breeding
makerules.tcl files.  Should that be here, or in a separate section?  If
the former, then should this section be renamed?


\section{\OOMMF\ Variable Types and Macros}\label{sec:vartypes}
{%
The following typedefs are defined in the
\fn{oommf/pkg/oc/{\it{platform}}/ocport.h} header file; this file is
created by the \app{pimake} build process (see
\fn{oommf/pkg/oc/procs.tcl}), and contains platform and machine
specific information.
\newcommand{\gbs}{\hspace{0.5em}}
\begin{itemize}
\item{\texttt{OC\_BOOL}} \gbs Boolean type, unspecified width.
\item{\texttt{OC\_BYTE}} \gbs Unsigned integer type exactly one byte wide.
\item{\texttt{OC\_CHAR}} \gbs Character type, may be signed or unsigned.
\item{\texttt{OC\_UCHAR}} \gbs Unsigned character type.
\item{\texttt{OC\_SCHAR}} \gbs Signed character type.  If \texttt{signed char}
  is not supported by a given compiler, then this falls back to a
  plain \texttt{char}, so use with caution.
\item{\texttt{OC\_INT2, OC\_INT4}} \gbs Signed integer with width of
  exactly 2, respectively 4, bytes.
\item{\texttt{OC\_INT2m, OC\_INT4m}} \gbs Signed integer with width of
  at least 2, respectively 4, bytes.  A type wider than the minimum
  may be specified if the wider type is handled faster by the
  particular machine.
\item{\texttt{OC\_UINT2, OC\_UINT4, OC\_UINT2m, OC\_UINT4m}} \gbs Unsigned
  integer versions of the preceding.
\item{\texttt{OC\_REAL4, OC\_REAL8}} \gbs Four byte, respectively eight
  byte, floating point variable.  Typically corresponds to \Cplusplus\
  ``float'' and ``double'' types.
\item{\texttt{OC\_REAL4m, OC\_REAL8m}} \gbs Floating point variable with
  width of at least 4, respectively 8, bytes.  A type wider than the minimum
  may be specified if the wider type is handled faster by the
  particular machine.
\item{\texttt{OC\_REALWIDE}} \gbs Widest type natively supported by the
  underlying hardware.  This is usually the \Cplusplus\ ``long
  double'' type, but may be overridden by the
\begin{center}
  \texttt{program\_compiler\_c++\_typedef\_realwide}
\end{center}
  option in the \fn{oommf/config/platform/{\it{platform}}.tcl} file.
\end{itemize}

The \fn{oommf/pkg/oc/{\it{platform}}/ocport.h} header file also
defines the following macros for use with the floating point variable
types:
\begin{itemize}
\item{\texttt{OC\_REAL8m\_IS\_DOUBLE}} \gbs True if \texttt{OC\_REAL8m} type
  corresponds to the \Cplusplus\ ``double'' type.
\item{\texttt{OC\_REAL8m\_IS\_REAL8}} \gbs True if \texttt{OC\_REAL8m} and
  \texttt{OC\_REAL8} refer to the same type.
\item{\texttt{OC\_REAL4\_EPSILON}} \gbs The smallest value that can be added to
  a \texttt{OC\_REAL4} value of ``1.0'' and yield a value different from
  ``1.0''.  For IEEE 754 compatible floating point, this should be
  \texttt{1.1920929e-007}.
\item{\texttt{OC\_SQRT\_REAL4\_EPSILON}}
    \gbs Square root of the preceding.
\item{\texttt{OC\_REAL8\_EPSILON}} \gbs The smallest value that can be added to
  a \texttt{OC\_REAL8} value of ``1.0'' and yield a value different from
  ``1.0''.  For IEEE 754 compatible floating point, this should be
  \texttt{2.2204460492503131e-016}.
\item{\texttt{OC\_SQRT\_REAL8\_EPSILON, OC\_CUBE\_ROOT\_REAL8\_EPSILON}}
    \gbs Square and cube roots of the preceding.
\item{\texttt{OC\_FP\_REGISTER\_EXTRA\_PRECISION}} \gbs True if
  intermediate floating point operations use a wider precision than
  the floating point variable type; notably, this occurs with some
  compilers on x86 hardware.
\end{itemize}

Note that all of the above macros have a leading ``\texttt{OC\_}''
prefix.  The prefix is intended to protect against possible name
collisions with system header files.  Versions of some of these macros
are also defined without the prefix; these definitions represent
backward support for existing \OOMMF\ extensions.  All new code
should use the versions with the ``\texttt{OC\_}'' prefix, and old code
should be updated where possible.  The complete list of deprecated
macros is:
\begin{quote}
\texttt{BOOL, UINT2m, INT4m, UINT4m,
    REAL4, REAL4m, REAL8, REAL8m, REALWIDE,
    REAL4\_EPSILON, REAL8\_EPSILON,
    SQRT\_REAL8\_EPSILON, CUBE\_ROOT\_REAL8\_EPSILON,
    FP\_REGISTER\_EXTRA\_PRECISION
}
\end{quote}

Macros for system identification:
\begin{itemize}
\item{\texttt{OC\_SYSTEM\_TYPE}} \gbs One of \texttt{OC\_UNIX} or
  \texttt{OC\_WINDOWS}.
\item{\texttt{OC\_SYSTEM\_SUBTYPE}} \gbs For unix systems, this is
    either \texttt{OC\_VANILLA} (general unix) or \texttt{OC\_DARWIN}
    (Mac OS X).  For Windows systems, this is generally
    \texttt{OC\_WINNT}, unless one is running out of a Cygwin shell,
    in which case the value is \texttt{OC\_CYGWIN}.
\end{itemize}

Additional macros and typedefs:
\begin{itemize}
\item{\texttt{OC\_POINTERWIDTH}} \gbs Width of pointer type, in bytes.
\item{\texttt{OC\_INDEX}} \gbs Typedef for signed array index type;
  typically the width of this (integer) type matches the width of the
  pointer type, but is in any event at least four bytes wide and not
  narrower than the pointer type.
\item{\texttt{OC\_UINDEX}} \gbs Typedef for unsigned version of
  OC\_INDEX.  It is intended for special-purpose use only.  In general,
  use OC\_INDEX where possible.
\item{\texttt{OC\_INDEX\_WIDTH}} \gbs Width of \texttt{OC\_INDEX} type.
\item{\texttt{OC\_BYTEORDER}} Either ``4321'' for little endian machines,
  or ``1234'' for big endian.
\item{\texttt{OC\_THROW(x)}} \gbs Throws a \Cplusplus\ exception with
  value ``x''.
\item{\texttt{OOMMF\_THREADS}} \gbs True threaded (multi-processing) builds.
\item{\texttt{OC\_USE\_NUMA}} \gbs If true, then NUMA (non-uniform memory
  access) libraries are available.
\end{itemize}
}

\section{\OOMMF\ eXtensible Solver}\label{sec:oxs}

The \OOMMF\ eXtensible Solver (OXS) top level architecture is shown in
\hyperrefhtml{the class diagram below}{Fig.~}{}{fig:oxsclass}.
The ``Tcl Control Script'' block represents the user interface and
associated control code, which is written in \Tcl.  The
micromagnetic problem input file is the content of the ``Problem
Specification'' block.  The input file should be a valid \MIF~2.0 file
(see the \OOMMF\ User's Guide for details on the \MIF\ file formats),
which also happens to be a valid \Tcl\ script.  The rest of the
architecture diagram represents \Cplusplus\ classes.

\ofig{\includeimage{6in}{!}{oxsclass}{OXS top-level class diagram}}{OXS
top-level class diagram.}{fig:oxsclass}

All interactions between the \Tcl\ script level and the core solver are
routed through the Director object.  Aside from the Director, all other
classes in this diagram are examples of \cd{Oxs\_Ext}
objects---technically, \Cplusplus\ child classes of the abstract
\cd{Oxs\_Ext} class.  OXS is designed to be extended primarily by the
addition of new \cd{Oxs\_Ext} child classes.

The general steps involved in adding an \cd{Oxs\_Ext} child class to OXS
are:
\begin{enumerate}
\item Add new source code files to \fn{oommf/app/oxs/local} containing
your class definitions.  The \Cplusplus\ non-header source code file(s)
must be given the \cd{.cc} extension.  (Header files are typically
denoted with the \cd{.h} extension, but this is not mandatory.)
\item Run \app{pimake} to compile your new code and link it in to the OXS
executable.
\item Add the appropriate \cd{Specify} blocks to your input \MIF~2.0
files.
\end{enumerate}
The source code can usually be modeled after an existing \cd{Oxs\_Ext}
object.  Refer to the Oxsii section of the \OOMMF\ User's Guide for a
description of the standard \cd{Oxs\_Ext} classes, or
\hyperrefhtml{below}{Sec.~}{}{sec:energyexample} for an annotated example of
an \cd{Oxs\_Energy} class.  Base details on adding a new energy term are
\hyperrefhtml{also presented below}{presented in Sec.~}{}{sec:energynew}.

The \app{pimake} application automatically detects all files in the
\fn{oommf/app/oxs/local} directory with the \cd{.cc} extension, and searches
them for \cd{\lb include} requests to construct a build dependency tree.
Then \app{pimake} compiles and links them together with the rest of the
OXS files into the \app{oxs} executable.  Because of the automatic file
detection, no modifications are required to any files of the standard
\OOMMF\ distribution in order to add local extensions.

Local extensions are then activated by \cd{Specify} requests in the
input \MIF~2.0 files.  The object name prefix in the \cd{Specify} block
is the same as the \Cplusplus\ class name.  All \cd{Oxs\_Ext} classes in
the standard distribution are distinguished by an \cd{Oxs\_} prefix.  It
is recommended that local extensions use a local prefix to avoid name
collisions with standard OXS objects.  (\Cplusplus\ namespaces are not
currently used in \OOMMF\ for compatibility with some older \Cplusplus\
compilers.)  The \cd{Specify} block initialization string format is
defined by the \cd{Oxs\_Ext} child class itself; therefore, as the
extension writer, you may choose any format that is convenient.
However, it is recommended that you follow the conventions laid out in
the \MIF~2.0 file format section of the \OOMMF\ User's Guide.


\subsection{Sample \cd{Oxs\_Energy} Class}\label{sec:energyexample}
This sections provides an extended dissection of a simple
\cd{Oxs\_Energy} child class.  The computational details are kept as
simple as possible, so the discussion can focus on the \Cplusplus\ class
structural details.  Although the calculation details will vary between
energy terms, the class structure issues discussed here apply across the
board to all energy terms.

The particular example presented here is for simulating
uniaxial magneto-crystalline energy, with a single anisotropy constant,
\cd{K1}, and a single axis, \cd{axis}, which are uniform across the
sample.
\begin{htmlonly}
The \htmlref{class definition}{fig:energyexampledfn} (.h) and
\htmlref{code}{fig:energyexamplecode} (.cc) files are included below.
\end{htmlonly}
\begin{latexonly}
The class definition (.h) and code (.cc) are displayed in
Fig.~\ref{fig:energyexampledfn} and \ref{fig:energyexamplecode},
respectively.
\end{latexonly}

\begin{codelisting}{p}{fig:energyexampledfn}{Example energy class
definition.}{sec:energyexample}
\begin{verbatim}
/* FILE: exampleanisotropy.h
 *
 * Example anisotropy class definition.
 * This class is derived from the Oxs_Energy class.
 *
 */

#ifndef _OXS_EXAMPLEANISOTROPY
#define _OXS_EXAMPLEANISOTROPY

#include "energy.h"
#include "threevector.h"
#include "meshvalue.h"

/* End includes */

class Oxs_ExampleAnisotropy:public Oxs_Energy {
private:
  double K1;        // Primary anisotropy coeficient
  ThreeVector axis; // Anisotropy direction
public:
  virtual const char* ClassName() const; // ClassName() is
  /// automatically generated by the OXS_EXT_REGISTER macro.
  virtual BOOL Init();
  Oxs_ExampleAnisotropy(const char* name,  // Child instance id
			Oxs_Director* newdtr, // App director
			Tcl_Interp* safe_interp, // Safe interpreter
			const char* argstr);  // MIF input block parameters

  virtual ~Oxs_ExampleAnisotropy() {}

  virtual void GetEnergyAndField(const Oxs_SimState& state,
                                 Oxs_MeshValue<REAL8m>& energy,
                                 Oxs_MeshValue<ThreeVector>& field
                                 ) const;
};


#endif // _OXS_EXAMPLEANISOTROPY
\end{verbatim}
\end{codelisting}

\begin{codelisting}{p}{fig:energyexamplecode}{Example energy class
code.}{sec:energyexample}
\begin{verbatim}
/* FILE: exampleanisotropy.cc            -*-Mode: c++-*-
 *
 * Example anisotropy class implementation.
 * This class is derived from the Oxs_Energy class.
 *
 */

#include "exampleanisotropy.h"

// Oxs_Ext registration support
OXS_EXT_REGISTER(Oxs_ExampleAnisotropy);

/* End includes */

#define MU0           12.56637061435917295385e-7   /* 4 PI 10^7 */

// Constructor
Oxs_ExampleAnisotropy::Oxs_ExampleAnisotropy(
  const char* name,     // Child instance id
  Oxs_Director* newdtr, // App director
  Tcl_Interp* safe_interp, // Safe interpreter
  const char* argstr)   // MIF input block parameters
  : Oxs_Energy(name,newdtr,safe_interp,argstr)
{
  // Process arguments
  K1=GetRealInitValue("K1");
  axis=GetThreeVectorInitValue("axis");
  VerifyAllInitArgsUsed();
}

BOOL Oxs_ExampleAnisotropy::Init()
{ return 1; }

void Oxs_ExampleAnisotropy::GetEnergyAndField
(const Oxs_SimState& state,
 Oxs_MeshValue<REAL8m>& energy,
 Oxs_MeshValue<ThreeVector>& field
 ) const
{
  const Oxs_MeshValue<REAL8m>& Ms_inverse = *(state.Ms_inverse);
  const Oxs_MeshValue<ThreeVector>& spin = state.spin;
  UINT4m size = state.mesh->Size();

  for(UINT4m i=0;i<size;++i) {
    REAL8m field_mult = (2.0/MU0)*K1*Ms_inverse[i];
    if(field_mult==0.0) {
      energy[i]=0.0;
      field[i].Set(0.,0.,0.);
      continue;
    }
    REAL8m dot = axis*spin[i];
    field[i] = (field_mult*dot) * axis;
    if(K1>0) {
      energy[i] = -K1*(dot*dot-1.0); // Make easy axis zero energy
    } else {
      energy[i] = -K1*dot*dot; // Easy plane is zero energy
    }
  }
}
\end{verbatim}
\end{codelisting}


\subsection{Writing a New \cd{Oxs\_Energy} Extension}\label{sec:energynew}
Under construction.








%\include{biblio}
%\bibliographystyle{osa}
\bibliographystyle{../common/oommf}
\bibliography{pmbiblio}

\section{Credits}\label{sec:credits}

\newcommand{\myCredit}[3]{\html{\htmladdnormallink{#1}{#2}}\latex{#1 (#3)}}


\newcommand{\CreditMJD}{%
\myCredit{Michael J. Donahue}{mailto:michael.donahue@nist.gov}{michael.donahue@nist.gov}}

\newcommand{\CreditDGP}{%
\myCredit{Donald G. Porter}{mailto:donald.porter@nist.gov}{donald.porter@nist.gov}}

\newcommand{\CreditRDM}{%
\myCredit{Robert D. McMichael}{mailto:rmcmichael@nist.gov}{rmcmichael@nist.gov}}

\newcommand{\CreditJE}{%
\myCredit{Jason Eicke}{mailto:jeicke@seas.gwu.edu}{jeicke@seas.gwu.edu}}

The main contributors to this document are \CreditMJD\ and \CreditDGP,
both of
\htmladdnormallink{ITL}{http://www.itl.nist.gov/}/\htmladdnormallink{NIST}{http://www.nist.gov/}.

If you have bug reports\index{reporting~bugs}, contributed code, feature
requests, or other comments for the \OOMMF\ developers, please send them
in an e-mail\index{e-mail} message to {\htmladdnormallink{{\tt
<michael.donahue@nist.gov>}}{mailto:michael.donahue@nist.gov}}%
\index{contact~information}.

% Give Latex2Html version and reference, as specified by the
% Perl $INFO variable set in .latex2html-init.
\htmlinfo*



\printindex

\end{document}
